\chapter{INTRODUÇÃO}
\label{sec:introdução}

Para inicializar o primeiro parágrafo de uma seção utilize o padrão a seguir. A introdução é a parte inicial do texto e que possibilita uma visão geral de todo o trabalho, devendo constar a delimitação do assunto tratado, objetivos da pesquisa, motivação para o desenvolvimento da mesma e outros elementos necessários para situar o tema do trabalho.

    Qual o problema que você está tentando resolver através do trabalho? Quais as restrições de projeto envolvidas?

    Nesta seção, você deve descrever a situação ou o contexto geral referente ao assunto em questão, devem constar informações atualizadas visando a proporcionar maior consistência ao trabalho.

\section{Objetivos}

    Os objetivos constituem a finalidade de um trabalho científico, ou seja, a meta que se pretende atingir com a elaboração da pesquisa.
    
    Podemos distinguir dois tipos de objetivos em um trabalho científico:
    
    \begin{itemize}
        \item Objetivos gerais – são aqueles mais amplos. São as metas de longo alcance, as contribuições que se desejam oferecer com a execução da pesquisa.

        \item Objetivos específicos – são a delimitação das metas mais específicas dentro do trabalho. São elas que, somadas, conduzirão ao desfecho do objetivo geral.
        
    \end{itemize}
    
    Como os objetivos indicam ação, recomenda-se que eles sejam definidos por meio de verbos, tais como analisar, avaliar, caracterizar, discutir, diagnosticar, investigar, implantar, pesquisar, realizar, determinar, etc.
    
\section{Organização do Trabalho}

    Nesta seção deve ser apresentado como está organizado o trabalho, sendo descrito, portanto, do que trata cada capítulo.

Nesta seção deve ser apresentado como está organizado o trabalho, sendo descrito, portanto, do que trata cada capítulo.

Nesta seção deve ser apresentado como está organizado o trabalho, sendo descrito, portanto, do que trata cada capítulo.

\subsection{Tamanho de Fonte para capítulo, seção e subseção}

    Elementos textuais

    - Fonte 12

    - Primária - caixa alta, com negrito
    
    - Secundária - caixa baixa, com negrito
    
    - Terciária - caixa baixa, sem negrito
    
    - Quaternária - caixa baixa, sem negrito
    
    - Espaçamento entre linhas simples.

Elementos pós-textuais

    - Fonte 12
    
    - Caixa alta
    
    - Com negrito.