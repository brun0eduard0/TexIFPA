\chapter{CAPÍTULO 3}
\label{sec:cap3}

Algumas regras devem ser observadas na redação da monografia:.

\begin{itemize}
    \item ser claro, preciso, direto, objetivo e conciso, utilizando frases curtas e evitando ordens inversas desnecessárias;
    
    \item construir períodos com no máximo duas ou três linhas, bem como parágrafos com cinco linhas cheias, em média, e no máximo oito (ou seja, não construir parágrafos e períodos muito longos, pois isso cansa o(s) leitor(es) e pode fazer com que ele(s) percam a linha de raciocínio desenvolvida);
    
    \item a simplicidade deve ser condição essencial do texto; a simplicidade do texto não implica necessariamente repetição de formas e frases desgastadas, uso exagerado de voz passiva (como será iniciado, será realizado), pobreza vocabular etc. Com palavras conhecidas de todos, é possível escrever de maneira original e criativa e produzir frases elegantes, variadas, fluentes e bem alinhavadas;
    
    \item adotar como norma a ordem direta, por ser aquela que conduz mais facilmente o leitor à essência do texto, dispensando detalhes irrelevantes e indo diretamente ao que interessa, sem rodeios (verborragias);

    \item não começar períodos ou parágrafos seguidos com a mesma palavra, nem usar repetidamente a mesma estrutura de frase;

    \item desprezar as longas descrições e relatar o fato no menor número possível de palavras;
    
    \item recorrer aos termos técnicos somente quando absolutamente indispensáveis e nesse caso colocar o seu significado entre parênteses (ou seja, não se deve admitir que todos os que lerão o trabalho já dispõem de algum conhecimento desenvolvido no mesmo);
    
    \item dispensar palavras e formas empoladas ou rebuscadas, que tentem transmitir ao leitor mera ideia de erudição;

    \item não perder de vista o universo vocabular do leitor, adotando a seguinte regra prática: nunca escrever o que não se diria;

    \item usar termos coloquiais ou de gíria com extrema parcimônia (ou mesmo nem serem utilizados) e apenas em casos muito especiais, para não darem ao leitor a ideia de vulgaridade e descaracterizar o trabalho;

    \item ser rigoroso na escolha das palavras do texto, desconfiando dos sinônimos perfeitos ou de termos que sirvam para todas as ocasiões;

    \item em geral, há uma palavra para definir uma situação;

    \item encadear o assunto de maneira suave e harmoniosa, evitando a criação de um texto onde os parágrafos se sucedem uns aos outros como compartimentos estanques, sem nenhuma fluência entre si;

    \item ter um extremo cuidado durante a redação do texto, principalmente com relação às regras gramaticais e ortográficas da língua;

    \item geralmente todo o texto é escrito na forma impessoal do verbo, não se utilizando, portanto, de termos em primeira pessoa, seja do plural ou do singular.

\end{itemize}

Continuação.

\section{Tabelas}

Teste de uma tabela:

\begin{lstlisting}[language=TeX, caption={Code Sem Sentido}]
\begin{Quadro}[H]
    \setstretch{1.5}\centering\footnotesize
    \caption{Quadro sem sentido.}
    \begin{center}
        \begin{tabular}{|c|c|}
            \hline
            \textbf{Título Coluna 1} & \textbf{Título Coluna 2}\\
            1 & 2 \\
            \hline
            X & Y \\
            \hline
            X & W \\
            \hline
        \end{tabular}
    \end{center}
    \label{qua:tabela-ssentido}
    \par Fonte: Elaborado pelo próprio autor.
\end{Quadro}
\end{lstlisting}


\begin{Quadro}[H]
    \setstretch{1.5}\centering\footnotesize
    \caption{Quadro sem sentido.}
    \begin{center}
        \begin{tabular}{|c|c|}
            \hline
            \textbf{Título Coluna 1} & \textbf{Título Coluna 2}\\
            1 & 2 \\
            \hline
            X & Y \\
            \hline
            X & W \\
            \hline
        \end{tabular}
    \end{center}
    \label{qua:tabela-ssentido}
    \par Fonte: Elaborado pelo próprio autor.
\end{Quadro}

\begin{lstlisting}[language=TeX, caption={Comunicação Code}]
\begin{Quadro}[H]
    \setstretch{1.5}\centering\footnotesize
    \caption{Comparação entre os padrões de comunicação sem fio.}
    \begin{center}
        \begin{tabular}{|c|c|c|}
            \hline
            \textbf{LAN sem fio} & \textbf{Frequência} & \textbf{Taxa de Transmissão}  \\ \hline
            {802.11b} & 2,4 a 2,485 GHz& 11 Mbps\\ \hline
            {802.11g} & 2,4 GHz e 5 GHz& 54 Mbps \\ \hline
            {802.11n} & 2,4 GHz e 5 GHz & 200 Mbps a 600 Mbps \\ \hline
            {LoRa} & 868, 433, 915, 902 e 928 MHz & 50 Kbps \\ \hline
        \end{tabular}
    \end{center}
    \label{qua:transmissaoSemFio}
    \par Fonte: Elaborado pelo próprio autor.
\end{Quadro}
\end{lstlisting}

\hfill

\begin{Quadro}[H]
    \setstretch{1.5}\centering\footnotesize
    \caption{Comparação entre os padrões de comunicação sem fio.}
    \begin{center}
        \begin{tabular}{|c|c|c|}
            \hline
            \textbf{LAN sem fio} & \textbf{Frequência} & \textbf{Taxa de Transmissão}  \\ \hline
            {802.11b} & 2,4 a 2,485 GHz& 11 Mbps\\ \hline
            {802.11g} & 2,4 GHz e 5 GHz& 54 Mbps \\ \hline
            {802.11n} & 2,4 GHz e 5 GHz & 200 Mbps a 600 Mbps \\ \hline
            {LoRa} & 868, 433, 915, 902 e 928 MHz & 50 Kbps \\ \hline
        \end{tabular}
    \end{center}
    \label{qua:transmissaoSemFio}
    \par Fonte: Elaborado pelo próprio autor.
\end{Quadro}

\begin{lstlisting}[language=TeX, caption={Cronograma Code}]
\begin{table}[H]
    \setstretch{1.5}\centering\footnotesize
    \caption{Cronograma}
    \begin{center}
        \begin{tabular}{p{0.8cm}S[table-format=0.1]S[table-format=0.1]S%
        [table-format=0.1]S[table-format=0.1]S[table-format=0.1]}
            \toprule
            {N\textdegree} & {Ano} & {Mula} & {\textit{Network Coding}} & %
            {\textit{Routing}} & {Simulador}
            \\
            \midrule
            {[1]} & {2012} & {Proteus} & {XOR} & \textit{store-and-forward}%
            & {aplicação real}   \\
            {[2]} & {2016} & {Ônibus} & {XOR} & \textit{store-and-forward}%
            & \textit{The ONE} \\ 
            {[3]} & {2016} & {Barco} & {XOR} & {ER, EOR, FCR, SWR} &%
            \textit{The ONE}   \\
            \bottomrule
        \end{tabular}
    \end{center}
    \label{table:cronogramaT}
    \par Fonte: Elaborado pelo próprio autor.
\end{table}
\end{lstlisting}


\begin{table}[H]
    \setstretch{1.5}\centering\footnotesize
    \caption{Cronograma}
    \begin{center}
        \begin{tabular}{p{0.8cm}S[table-format=0.1]S[table-format=0.1]S%
        [table-format=0.1]S[table-format=0.1]S[table-format=0.1]}
            \toprule
            {N\textdegree} & {Ano} & {Mula} & {\textit{Network Coding}} & %
            {\textit{Routing}} & {Simulador}
            \\
            \midrule
            {[1]} & {2012} & {Proteus} & {XOR} & \textit{store-and-forward}%
            & {aplicação real}   \\
            {[2]} & {2016} & {Ônibus} & {XOR} & \textit{store-and-forward}%
            & \textit{The ONE} \\ 
            {[3]} & {2016} & {Barco} & {XOR} & {ER, EOR, FCR, SWR} &%
            \textit{The ONE}   \\
            \bottomrule
        \end{tabular}
    \end{center}
    \label{table:cronogramaT}
    \par Fonte: Elaborado pelo próprio autor.
\end{table}


\begin{lstlisting}[language=TeX, caption={Trabalhos NC}]
\begin{table}[H]
    \setstretch{1.5}\centering\footnotesize
    \caption{Relação dos trabalhos }	
    \begin{center}
        \begin{tabular}{p{0.8cm}S[table-format=0.1]S[table-format=0.1]S%
        [table-format=0.1]S[table-format=0.1]S[table-format=0.1]}
            \toprule
            {N\textdegree} & {Ano} & {Mula} & {\textit{Network Coding}} & %
            {\textit{Routing}} & {Simulador}
            \\
            \midrule
            {[1]} & {2012} & {Proteus} & {XOR} & \textit{store-and-forward}%
            & {aplicação real}   \\
            {[2]} & {2016} & {Ônibus} & {XOR} & \textit{store-and-forward}%
            & \textit{The ONE} \\ 
            {[3]} & {2016} & {Barco} & {XOR} & {ER, EOR, FCR, SWR} &%
            \textit{The ONE}   \\
            \bottomrule
        \end{tabular}
    \end{center}
    \label{table:trabalhoMulaNC}
    \par Fonte: Elaborado pelo próprio autor.
\end{table}
\end{lstlisting}

\hfill

\begin{table}[H]
    \setstretch{1.5}\centering\footnotesize
    \caption{Relação dos trabalhos }	
    \begin{center}
        \begin{tabular}{p{0.8cm}S[table-format=0.1]S[table-format=0.1]S%
        [table-format=0.1]S[table-format=0.1]S[table-format=0.1]}
            \toprule
            {N\textdegree} & {Ano} & {Mula} & {\textit{Network Coding}} & %
            {\textit{Routing}} & {Simulador}
            \\
            \midrule
            {[1]} & {2012} & {Proteus} & {XOR} & \textit{store-and-forward}%
            & {aplicação real}   \\
            {[2]} & {2016} & {Ônibus} & {XOR} & \textit{store-and-forward}%
            & \textit{The ONE} \\ 
            {[3]} & {2016} & {Barco} & {XOR} & {ER, EOR, FCR, SWR} &%
            \textit{The ONE}   \\
            \bottomrule
        \end{tabular}
    \end{center}
    \label{table:trabalhoMulaNC}
    \par Fonte: Elaborado pelo próprio autor.
\end{table}

\section{Seção 2}

Seção 2.
