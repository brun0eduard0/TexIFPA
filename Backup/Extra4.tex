\chapter{CITAÇÕES}
\label{sec:citacoes}

\section{Citações}
    \subsection{Citações com autoria dada pelo título (cite) }

    Padrão de referência para citação implícita e explícita.
    
    Artigo \cite{negrão14}
    
    Livro \cite[p.~55]{higginbottom1998performance}.
    
    Livro \cite{kurose2013computer}.
    
    Relatório Técnico \cite{Scott2007}.
    
    Site \cite{Raspberry}.

\begin{lstlisting}[language=TeX, caption={citação1}]
Artigo \cite{negrão14}

Livro \cite[p.~55]{higginbottom1998performance}.

Livro \cite{kurose2013computer}.

Relatório Técnico \cite{Scott2007}.

Site \cite{Raspberry}.    
\end{lstlisting}

   \subsection{Referências citadas por outra fonte (apud)}
    
    Livro \apud{higginbottom1998performance}{kurose2013computer}.
    
    Relatório Técnico \apudonline{Scott2007}{negrão14}.

\begin{lstlisting}[language=TeX, caption={citação2}]
Livro \apud{higginbottom1998performance}{kurose2013computer}.

Relatório Técnico \apudonline{Scott2007}{negrão14}.
\end{lstlisting}

   \subsection{Citações Múltiplas}
    
    Artigo \cite{negrão14, higginbottom1998performance, kurose2013computer}.
    
    Relatório Técnico \citeonline{Scott2007, kurose2013computer}.

\begin{lstlisting}[language=TeX, caption={citação4}]
Artigo \cite{negrão14, higginbottom1998performance, kurose2013computer}.

Relatório Técnico \citeonline{Scott2007, kurose2013computer}.
\end{lstlisting}

\section{Referenciar seção, imagem, tabela}

    Use o barra \ acompanhado do ref e logo depois adicione o tipo:referencia. Referência da imagem da logo da UFAM \ref{fig:logoIFPA} acompanhada da seção de introdução \ref{sec:introdução} na página \pageref{fig:logoIFPA}.

\begin{lstlisting}[language=TeX, caption={Refenciando Elementos}]
Referenciando imagem  \ref{fig:logoIFPA}

Referenciando seção \ref{sec:introdução}

Referenciando página \pageref{fig:logoIFPA}
\end{lstlisting}

\section{Adicionando Código}

Para adicionar código usando a biblioteca lstlisting, você deverá recorrer ao seguinte padrão.

\begin{lstlisting}[language=TeX, caption={Adicionando Código em LaTeX}]
\begin{lstlisting}[language=TeX, caption={Adicionando Código em LaTeX}]

Código em Latex com citação \cite{kurose2013computer}.

\end{ lstlisting}    %%% SEM O ESPAÇO EM BRANCO
\end{lstlisting}

\begin{lstlisting}[language=TeX, caption={Adicionando Código - Como aparecerá}]

Código em Latex com citação \cite{kurose2013computer}.

\end{lstlisting}

\begin{lstlisting}[language=C, caption={HelloWorld.c - neste template}]
\begin{lstlisting}[language=C, caption={HelloWorld.c - neste template}]
#include<stdio.h>

void main(){
    printf("Ola Mundo");
}
\end{ lstlisting}    // SEM O ESPAÇO EM BRANCO
\end{lstlisting}

\begin{lstlisting}[language=C, caption={HelloWorld.c - Como aparecerá}]
#include<stdio.h>

void main(){
    printf("Ola Mundo");
}
\end{lstlisting}