% ---
% Pacotes básicos
% ---
\usepackage{bookmark}       % Usa a fonte Bookman Old Style
\usepackage[utf8]{inputenc} % permite edição direta com acentos
\usepackage[brazil]{babel}  % hifenização e títulos em português do Brasil
\usepackage[T1]{fontenc}	% Selecao de codigos de fonte.
\usepackage{color}			% Controle das cores
\usepackage{helvet}
\renewcommand{\familydefault}{\sfdefault}
\usepackage{float}
\usepackage{xurl}
\usepackage{url}
\usepackage{enumerate}
\usepackage[euler]{textgreek}
\usepackage{pdfpages}
\usepackage{csquotes}
\usepackage{lipsum}
\usepackage{textcomp}
\usepackage{parskip}
\DisemulatePackage{setspace}

%%%    IMAGENS
\usepackage{array}
\usepackage{graphicx}		% Inclusão de gráficos
\graphicspath{{imagens/}}

%%%    TABELAS E QUADRO
\usepackage{tabularx,ragged2e}	% Para inserir tabelas
\usepackage{multirow}			% Para mesclar células
\usepackage[dvipsnames,table,xcdraw]{xcolor}		% Permite adicionar cores nas linhas de tabelas
\usepackage{amsfonts}			% Permite usar notação de conjuntos
\usepackage{amsmath}			% Permite citar equações
\usepackage{amsthm}				% Permite criar teoremas e experimentos
%\usepackage[font={bf, small}, labelsep=endash, labelfont=bf]{caption}	% Faz legenda de figuras ficarem em negrito
\usepackage[font={small}, labelsep=endash]{caption}	% Faz legenda de figuras ficarem em negrito


% -------------------------------------------------
% Pacotes para QUDROS    -     funciona
% -------------------------------------------------
%\newcommand{\autodot}{}%comando criado para anular o erro
%\usepackage{tocbibind}
%\usepackage{tocbasic}
%\DeclareTOCStyleEntries{tocline}{figure,table}% now controlled by tocbasic
%\DeclareNewTOC[%
%    type=quadro,%
%    float, % define a floating environment
%    name=Quadro, %
%    listname={LISTA DE QUADROS}, %
%    setup=totoc,    % entry in ToC
%    tocentryindent:=figure,% same indentation as for figure
%    tocentrynumwidth:=figure,% same numwidth as for figure
%    floattype=4,%
%]{loq}
%\setuptoc{loq}{chapteratlist}
%%%Colocando este comando resolve o problema da lista contudo cria outro erro

\usepackage{cancel}				% Permite fazer expressão tendendo a zero
\usepackage{epstopdf}			% Converte eps para pdf

%%%    índex
\usepackage[imakeidx]{knowledge}
\indexsetup{othercode=\small}
\makeindex[program=makeindex,columns=1,intoc=true,options={-s 0_prepos/indexStyle.ist}]
\makeatletter
\def\@idxitem{\par\hangindent 0pt}
\makeatother

%%%%%%%%% NOVO
\usepackage{microtype} 	% para melhorias de justificação
\usepackage{setspace}
\urlstyle{same}
\usepackage{indentfirst} % Indenta o primeiro parágrafo de cada seção.

%%%    tabelas - configuração
\usepackage{booktabs}
\usepackage{siunitx}
\usepackage{lscape}

%%%     hifenização do português para o preambulo
\usepackage{geometry}
\usepackage{hyphenat}
\tolerance=9000

%%% lista de abreviações e siglas automática 
% -------------------------------------------------
% CONFIGURAÇÃO DE CAPÍTULOS E SEÇÕES
% -------------------------------------------------
\usepackage{titlesec}
% --------- Capítulo
\titleformat{\chapter}
    {\centering\normalfont\sffamily\bfseries\MakeUppercase}{\thechapter}{0em}{}
\titlespacing{\chapter}{0em}{0em}{3em}
% --------- Seção
\titleformat{\section}
    {\normalfont\sffamily\normalsize\bfseries}{\thesection}{0.5em}{}
\titlespacing{\section}{0em}{1.5em}{1.5em}
% --------- Subseção
\titleformat{\subsection}
    {\normalfont\sffamily\normalsize}{\thesubsection}{0.5em}{}
\titlespacing{\subsection}{0em}{1.5em}{0em}

% ------------------------------------------------
% Glossário, Index
% -------------------------------------------------
\usepackage[
    record=hybrid,
    symbols,
    abbreviations,
    acronym,
    toc,
    automake,
    xindy={language=portuguese},
    postdot,
    nonumberlist
]{glossaries-extra}
\usepackage{glossary-longbooktabs}
\usepackage{glossaries-extra-stylemods}
\usepackage{glossary-inline}
\setglossarystyle{list}
\glssetcategoryattribute{general}{indexname}{textbf}
\glssetcategoryattribute{general}{dualindex}{true}
\glssetcategoryattribute{abbreviation}{indexname}{textbf}
\glssetcategoryattribute{abbreviation}{dualindex}{true}
\glssetcategoryattribute{general}{glossname}{firstuc}
\GlsXtrEnableIndexFormatOverride
\renewcommand*{\glsxtrautoindexentry}[1]{\string\glsentryfirst{#1}}
\renewcommand*{\glsxtrautoindexassignsort}[2]{%
  \ifglshaslong{#2}%
  {\glsletentryfield{#1}{#2}{long}}%
  {\glsletentryfield{#1}{#2}{sort}}%
}
\makeglossaries
\renewcommand*{\glsnamefont}[1]{%
    \textmd{#1}:%
}
% PADRÃO DE ENTRADA NO GLOSSÁRIO
%\newglossaryentry{aPalavra}{
%        name= {APALVRA},       %%% EM CAIXA ALTA, É O QUE IRÁ APARECER NO GLOSSÁRIO, PARA SEGUIR A NORMA DO IFPA
%        text = {aPalavra},    %%% como irá aparecer no TEXTO
%        sort = {Formula},     %% REESCREVA A PALAVRA SEM USAR NENHUMA ACENTUAÇÃO PARA QUE POSSA SER USADA PARA ORGANZIAR O GLOSSÁRIO
%        description={É uma linguagem de marcação especialmente feita para documentos científicos}              %DESCRIÇÃO
%}
%
% ADICIONE AS ENTRADAS USANDO A ORDEM ALFABÉTICA
%

\newglossaryentry{fórmula}{
        name = {FÓRMULA},
        text = {fórmula},
        sort = {Formula},     %% PALAVRA ACENTUADAS DEVERÃO CONTER O CAMPO 'SORT' PARA AJUDAR NA ORDENAÇÃO 
        description={Uma expressão matemática}
}

\newglossaryentry{latex}{
        name= {LATEX},
        text = {latex},
        sort = {latex},
        description={É uma linguagem de marcação especialmente feita para documentos científicos}
}

\newglossaryentry{matemática}{
        name = {MATEMÁTICA},
        text = {matemática},
        sort = {matematica},
        description={Matemática é o que a matemática faz}
}

\longnewglossaryentry{sample}{
    name = {SAMPLE},
    text = {sample},            % -    versão em texto
    plural = {samples},         %  -    versão em plural no texto
    sort = {sample},
}{%
     {an example}       %DESCRIÇÃO
}

%\newterm[〈key=value list〉]{〈entry-label〉}
%This is just a shortcut that uses \newglossaryentry with the name set to 〈entry-label〉 and
%the description is suppressed.


%\newglossaryentry{not:set}
%{% This entry goes in the `notation' glossary:
%type={notation},
%name={$\mathcal{S}$},
%text={\mathcal{S}},
%description={A set},
%sort={S}
%}

% -------------------------------------------------
% abreviações e siglas
% -------------------------------------------------
\usepackage{acro}
%%%  LISTA DE ABREVIAÇÕES e SIMBOLOS
% no decorrer da sua escrita, você usará abreviações e símbolos
% estes simbolos e abreviações devem constar na lista de abreviações
% e simbolos. Para gerar AUTOMATICAMENTE estas listas sem o menor esforço 
% você deve declarar NESTE documento todas as ABREVIAÇÕES e todos os SÍMBOLOS
% que você usará no decorrer do texto. 
% 
% SOMENTE DEPOIS DE DECLARAR AQUI
%
% você será capaz de pôr o simbolo usando a biblioteca ACRONYM
%
% usando os comandos        \ac      e     \acl
%	
%	 Para utilizar abreviaturas e simbolos, basta utilizar o comando barra \ combinado com ac ou acl.
%	 
%	 Abreviaturas e a versão por extenso.
%	 
%	 AC \ac{abntex}.     MOSTRARÁ A ABREVIAÇÃO
%%	 
%	 ACL \acl{abntex}.   MOSTRARÁ AS PALAVRAS DA ABREVIAÇÃO POR EXTENSO
%	 
%%	 Simbolos e a versão por extenso.
%	 
%	 AC  \ac{lambda}.   MOSTRARÁ O SIMBOLO
%	 
%	 ACL \acl{lambda}.  MOSTRARÁ A PALAVRA DO SIMBOLO POR EXTENSO
% 
% class `abbrev': abbreviations:
%NOTE QUE SOMENTE OS QUE SÃO CHAMADOS NO DECORRER DO TEXTO IRÃO APARECER NA LISTA DE ABREVIATURAS E SIMBOLOS
%PADRÃO PARA ABREVIATURA

%\DeclareAcronym{}{
%	short = \normalfont{},
%	long  = \textit{},
%	class = abbrev
%}

\DeclareAcronym{sbc}{
	short = \normalfont{SBC},
	short-plural = s,
	long  = Sociedade Brasileira de Computação,
	tag = abbrev
}

\DeclareAcronym{nc}{
	short = \normalfont{NC},
	long  = \textit{Networking Coding},
	tag = abbrev
}

\DeclareAcronym{abnt}{
	short = \normalfont{ABNT},
	short-plural = s,
	long  = Associação Brasileira de Normas Técnicas,
	tag = abbrev
}

\DeclareAcronym{abntex}{
	short = \normalfont{abnTeX},
	short-plural = s,
	long  = ABsurdas Normas para TeX,
	tag = abbrev
}


%%%  LISTA DE SIMBOLOS
%PADRÃO PARA SIMBOLO
%\DeclareAcronym{}{
%	short = \normalfont{},
%	long  = \textit{},
%   sort = ,
%	tag = nomen
%}

\DeclareAcronym{megaByte}{
	short = \normalfont{MB},
	long  = é uma unidade de medida de informação que equivale a 1000000 bytes,
	sort  = MB,
	tag = nomen,
}

\DeclareAcronym{lambda}{
	short = \normalfont{\textlambda} ,
	long  = comprimento de onda,
	sort  = \textlambda,
	tag = nomen,
}

% -------------------------------------------------
%%%%%% estilo do capítulo
%%%%%% escolha 1 e tire o comentário
%% acesse a página e escolha
%%http://ctan.math.washington.edu/tex-archive/info/latex-samples/MemoirChapStyles/MemoirChapStyles.pdf
% -------------------------------------------------
\setlength\midchapskip{12pt}
%%%%%%%%   CAPA
\instituicao{INSTITUTO FEDERAL DE EDUCAÇÃO, CIÊNCIA E TECNOLOGIA DO PARÁ}

%%%%%% caso esteja envolvido em algum plano ou programa de pós-graduação de curso específico, 
%% preencha o comando abaixo
\plano{}

\campus{CAMPUS ITAITUBA}

\curso{TECNÓLOGO EM ANÁLISE E DESENVOLVIMENTO DE SISTEMAS}

%(use \\ para separar os autores)
\autor{
    NOME DOS ALUNOS EM CAIXA ALTA
    \\
    OUTRO FULANO DE TAL DA CASA QUEBRADA
    \\
    OUTRO FULANO DE TAL DA CASA ARRUMADA
    \\
    NOME DO OUTRA AUTOR EM CAIXA ALTA AO QUADRADO
    \\
    OUTRO FULANO DE TAL DA CASA BONITA
}

\titulo{NOME DA TESE, DISSERTAÇÃO OU TCC EM CAIXA ALTA}

\tipotrabalho{Monografia}

\local{ITAITUBA}

\data{2023}

%%%%%%%%   CONTRACAPA

\orientador[Orientador(a)]{Prof.\textdegree \space Dr. Fulano de Tal}%Nome e titulação do(a) professor(a) orientador(a)}

%%%% COLOQUE O MESMO TEXTO EM AMBOS

% O preambulo deve conter o tipo do trabalho, o objetivo, 
% o nome da instituição e a área de concentração 

\hyphenation{
    Trabalho de conclusão de curso apresentado ao Instituto Federal de Ciência e Tecnologia do Pará - IFPA Campus Itaituba como requisito para obtenção de grau em Tecnólogo em Análise e Desenvolvimento de Sistemas.
}
\preambulo{
    Trabalho de conclusão de curso apresentado ao Instituto Federal de Ciência e Tecnologia do Pará - IFPA Campus Itaituba como requisito para obtenção de grau em Tecnólogo em Análise e Desenvolvimento de Sistemas.
}

% -------------------------------------------------
% Informações do PDF
% -------------------------------------------------
\makeatletter
\hypersetup{
    colorlinks=true,
    linkcolor=blue,
    citecolor=blue,
    filecolor=magenta,      
    urlcolor=blue,
    bookmarksopen=true,
    %pagebackref=true,
	pdftitle={\imprimirtitulo}, 
	pdfauthor={\imprimirautor},
    pdfsubject={\imprimirpreambulo},
    pdfcreator={LaTeX with abnTeX2},
	pdfkeywords={\imprimirpalavraschave}{trabalho acadêmico},
	bookmarksdepth=4,
    breaklinks=true,
}
\makeatother

% -------------------------------------------------
% Espaçamentos entre linhas e parágrafos 
% -------------------------------------------------
% O tamanho do parágrafo é dado por:
\setlength{\parskip}{0.0cm}
\setlength{\parindent}{1.0cm}
\renewcommand{\baselinestretch}{1.5}
\renewcommand*{\afterchapskip}{1.5ex}

% -------------------------------------------------
% Referências Bibliográficas se dá pela pasta abnt
% e a seguinte biblioteca
% -------------------------------------------------
\usepackage{hyperref}
\usepackage[brazilian,hyperpageref]{backref}	 % Paginas com as citações na bibl
\usepackage{breakurl}
\usepackage[
        versalete,
        abnt-emphasize = bf, % destaca o titulo da revista ou livro em negrito;
        abnt-etal-list = 3, % trabalhos com mais de 3 autores recebem et al.,;
        abnt-etal-text = it, % escreve o et al., em italico;
        abnt-and-type = &, % usa o carater '&' no lugar de 'e' para mais de um autor;
        abnt-last-names = abnt, % trata sobrenomes 'estritamente' conforme a ABNT; e
        abnt-repeated-author-omit = no, % autores com + de uma entrada recebem '____.',
        alf
%        abnt-url-package = url
        ]{abntex2cite}
\usepackage{Sys/0_prepos/url6023}

% -------------------------------------------------
% Biblioteca para adicionar CÓDIGO
% -------------------------------------------------
\usepackage{listings}
\usepackage{xcolor}
\definecolor{codegreen}{rgb}{0,0.6,0}
\definecolor{codegray}{rgb}{0.5,0.5,0.5}
\definecolor{codepurple}{rgb}{0.58,0,0.82}
\definecolor{backcolour}{rgb}{0.95,0.95,0.92}
\lstdefinestyle{mystyle}{
    backgroundcolor=\color{backcolour},   
    commentstyle=\color{codegreen},
    keywordstyle=\color{magenta},
    numberstyle=\tiny\color{codegray},
    stringstyle=\color{codepurple},
    basicstyle=\ttfamily\footnotesize,
    breakatwhitespace=true,
    escapeinside={\%*}{*)},  % if you want to add LaTeX within your code
    keepspaces=true,  % keeps spaces in text
    mathescape=true,
    breaklines=true,
    caption={Código},
    captionpos=t,
    numbers=left,                    
    numbersep=2pt,                  
    stepnumber=1,
    numberstyle=\tiny,
    numbersep=10pt,
    showstringspaces=false,
    showtabs=false,                  
    tabsize=2,
    extendedchars=true,
    inputencoding=utf8,
    literate={á}{{\'a}}1  {é}{{\'e}}1  {í}{{\'i}}1 {ó}{{\'o}}1  {ú}{{\'u}}1 {Á}{{\'A}}1  {É}{{\'E}}1  {Í}{{\'I}}1 {Ó}{{\'O}}1  {Ú}{{\'U}}1 {à}{{\`a}}1  {è}{{\`e}}1  {ì}{{\`i}}1 {ò}{{\`o}}1  {ù}{{\`u}}1 {À}{{\`A}}1  {È}{{\'E}}1  {Ì}{{\`I}}1 {Ò}{{\`O}}1  {Ù}{{\`U}}1 {ä}{{\"a}}1  {ë}{{\"e}}1  {ï}{{\"i}}1 {ö}{{\"o}}1  {ü}{{\"u}}1 {Ä}{{\"A}}1  {Ë}{{\"E}}1  {Ï}{{\"I}}1 {Ö}{{\"O}}1  {Ü}{{\"U}}1 {â}{{\^a}}1  {ê}{{\^e}}1  {î}{{\^i}}1 {ô}{{\^o}}1  {û}{{\^u}}1 {Â}{{\^A}}1  {Ê}{{\^E}}1  {Î}{{\^I}}1 {Ô}{{\^O}}1  {Û}{{\^U}}1 {œ}{{\oe}}1  {Œ}{{\OE}}1  {æ}{{\ae}}1 {Æ}{{\AE}}1  {ß}{{\ss}}1 {ç}{{\c c}}1 {Ç}{{\c C}}1 {ø}{{\o}}1  {Ø}{{\O}}1   {å}{{\r a}}1 {Å}{{\r A}}1 {ã}{{\~a}}1  {õ}{{\~o}}1 {Ã}{{\~A}}1  {Õ}{{\~O}}1 {ñ}{{\~n}}1  {Ñ}{{\~N}}1  {¿}{{?`}}1  {¡}{{!`}}1 {°}{{\textdegree}}1 {º}{{\textordmasculine}}1 {ª}{{\textordfeminine}}1,
    frame=single,
    framerule=0pt,
    framesep=-2pt,
    rulesep=0pt,
    rulecolor=\color{black},
    showspaces=false,
    %nolol=false,
    numberbychapter=false,
}
\lstset{style=mystyle}
\usepackage[hypcap]{caption}
\DeclareCaptionLabelFormat{bf-parens}{(\textbf{#2})}
\captionsetup[lstlisting]{singlelinecheck=false, labelsep=endash}
\renewcommand\lstlistingname{Código}
\renewcommand\lstlistlistingname{Lista de Códigos}


% -------------------------------------------------
% Padrão para CAPA e CONTRACAPA
% -------------------------------------------------
\geometry{a4paper, left=3cm, right=2cm, top=3cm, bottom=2cm, headheight=0.5cm, textheight=44\baselineskip, headsep=0.01cm}

% -------------------------------------------------
% Pacotes para sumário
% -------------------------------------------------
\usepackage{tocloft}

%8888888888888888888888888888888888888888888888888888
%            ACRESCENTADOS POR ELIESIO
%8888888888888888888888888888888888888888888888888888
\usepackage{tcolorbox}
\usepackage{pifont}
\usepackage{tikz}
\newcommand{\ajuda}{
\begin{tcolorbox}[colback=blue!5!white,colframe=red!25!blue,title= Comando de Ajuda]
Digite os comandos em \textcolor{red}{Vermelho} para obter a ajuda desejada.\\[5pt]
\textcolor{red}{$\backslash$estruturaDoc} Para ajudá-lo com a estrutura do seu documento; \\[5pt]
\textcolor{red}{$\backslash$upArquivos} Para ajudá-lo com o Upload de arquivos para o overleaf; \\[5pt]
\textcolor{red}{$\backslash$fazerLista} Para ajudá-lo a  fazer uma lista numerada ou não; \\\textcolor{red}{$\backslash$adicionaFormula} Para ajudá-lo a adicionar uma equação matemática numerada ou não; \\[5pt]
\textcolor{red}{$\backslash$adicionaCodigo} Para ajudá-lo na adição de um Código;\\
\textcolor{red}{$\backslash$adicionaQuadro} Para ajudá-lo na adição de um Quadro;\\
\textcolor{red}{$\backslash$adicionaTabela} Para ajudá-lo na adição de uma Tabela;\\
\textcolor{red}{$\backslash$adicionaFig} Para ajudá-lo na adição de uma figura; \\[5pt]
\textcolor{red}{$\backslash$letrasGregas} Para ajudá-lo na adição de caracteres gregos;\\
\textcolor{red}{$\backslash$caracterEspecial} Para ajudá-lo na adição de caracteres especiais;\\
\textcolor{red}{$\backslash$adicionaSigla} Para ajudá-lo na adição de uma Sigla;\\
\textcolor{red}{$\backslash$adicionaSimbolo} Para ajudá-lo na adição de um Simbolo;\\
\textcolor{red}{$\backslash$adicionaPalavraGlossario} Para ajudá-lo na adição de uma Palavra ao Glossário;\\[10pt]
Quando não precisar mais de ajuda, comente a linha \textcolor{blue}{$\backslash$ajuda} \texttt{ - para comentar digite o caracter \% antes do comando. Ex. \textcolor{black!40!white}{\%$\backslash$ajuda}}
\end{tcolorbox}
}

\newcommand{\adicionaFormula}{
\begin{tcolorbox}[colback=blue!5!white,colframe=black!25!pink,title= Comandos básicos utilizados para acrescentar equações numeradas ou não]
\ding{42} Formas de adicionar uma equação não numerada:\\[5pt]
\ding{172} No meio de texto; 
$$\text{texto1} \$\texttt{equação}\$\text{ texto2}$$
Ex. O teorema \$c\^{}2 = a\^{}2 + b\^{}2\$ a seguir $\Rightarrow$  O teorema $c^2 = a^2 + b^2$ a seguir\\[5pt]
\ding{173} Centralizado em linha separada
$$\$\$\text{ equação }\$\$$$
Ex. \$\$\textcolor{blue}{$\backslash$displaystyle $\backslash$sum}\_\{i=1\}\^{}n f\textcolor{blue}{$\backslash$left}(x\_i\textcolor{blue}{$\backslash$right})\$\$ quando compilado
$$\displaystyle \sum_{i=1}^nf\left(x_i\right)$$
\ding{42} Formas de adicionar uma equação numerada:\\[5pt]
\textcolor{blue}{$\backslash$begin}\{equation\} \texttt{ - inicia o ambiente equação}\\
\textcolor{blue}{$\backslash$displaystyle $\backslash$int}\_a\^{}b f(x)dx = F(b) - F(a) \texttt{ - equação}\\
\textcolor{blue}{$\backslash$label}\{eq:cap1eq1\} \texttt{ - rótulo da equação no documento}\\
\textcolor{blue}{$\backslash$end}\{equation\} \texttt{ - finaliza o ambiente equação}\\[10pt]
Quando compilada a equação \ref{eq:cap1eq1} ficaria
\begin{equation}
    \displaystyle \int_a^bf(x)dx=F(b)-F(a)
    \label{eq:cap1eq1}
\end{equation}
\ding{96} necessário no caso de referenciar em outra parte do documento\\
Para não exibir numeração \textcolor{blue}{$\backslash$begin}\{equation*\} $\ldots$ \textcolor{blue}{$\backslash$end}\{equation*\}\\[5pt]
Para a equação à esquerda modifique as configurações no preambulo (antes dos pacotes) \textcolor{blue}{$\backslash$documentclass}[fleqn]\{classe do documento\}
\end{tcolorbox}
}

\newcommand{\estruturaDoc}{
\begin{tcolorbox}[colback=blue!5!white,colframe=black!25!yellow,title= Comandos básico utilizados na estrutura desse documento]
Principais comando utilizados na estrutura hierárquica desse documento:\\[5pt]
\textcolor{blue}{$\backslash$chapter}\{título do capítulo\}\texttt{ - para abreviar no sumário utilize ``$\backslash$chapter[título abreviado]\{título do capítulo\}'' sem aspas}\\
\textcolor{blue}{$\backslash$section}\{título da seção\}\texttt{ - texto do segundo nível}\\
\textcolor{blue}{$\backslash$subsection}\{título da subseção\}\texttt{ - texto do terceiro nível}\\
\textcolor{blue}{$\backslash$subsubsection}\{título do 4\textdegree~ nível\}\\[10pt]
Capítulos, seções, subseções e subsubseções podem ser rotulados, logo abaixo da entrada do capítulo, para eventuais chamadas em outras partes do documento.\\[10pt]
\textcolor{blue}{$\backslash$label}\{sec:introdução\}\texttt{ - exemplo de entrada do rótulo de chamada da introdução}\\[10pt]
\textcolor{blue}{$\backslash$ref}\{sec:introdução\} \texttt{ - quando digitado no documento, faz a referência ao capítulo introdução} 
\end{tcolorbox}
}
\newcommand{\caracterEspecial}{
\begin{tcolorbox}[colback=blue!5!white,colframe=red!25!yellow,title= Comandos para caracteres especiais]
Para adicionar um caracter especial digite um dos comando em \textcolor{blue}{azul} abaixo:\\[5pt]
\begin{tabular}{lclp{3cm}rcl}
\textcolor{blue}{$\backslash$\$}&\texttt{-  para -}&\$&&\textcolor{blue}{$\backslash$\&}&\texttt{-  para -}&\&\\
\textcolor{blue}{$\backslash$\%}&\texttt{-  para -}&\%&&\textcolor{blue}{$\backslash$\#}&\texttt{-  para -}&\#\\
\textcolor{blue}{$\backslash$\_}&\texttt{-  para -}&\_&&\textcolor{blue}{$\backslash$\{}&\texttt{-  para -}&\{\\
\textcolor{blue}{$\backslash$\}}&\texttt{-  para -}&\}&&\textcolor{blue}{$\backslash$\~{}\{\}}&\texttt{-  para -}&\~{}\\
\textcolor{blue}{$\backslash$\^{}\{\}}&\texttt{-  para -}&\^{}&&\textcolor{blue}{\$$\backslash$backslash\$}&\texttt{-  para -}&$\backslash$\\
\end{tabular}
\end{tcolorbox}
}

\newcommand{\letrasGregas}{
\begin{tcolorbox}[colback=blue!5!white,colframe=red!75!yellow,title= Comandos para caracteres gregos]
Para adicionar uma letra grega digite um dos comando em \textcolor{blue}{azul} abaixo:\\[5pt]
\begin{tabular}{lcl}
\textcolor{blue}{$\backslash$alpha} \texttt{-  para a letra \alpha}&&\textcolor{blue}{$\backslash$beta} \texttt{-  para a letra \beta}\\
\textcolor{blue}{$\backslash$gamma} \texttt{-  para a letra \gamma}&&\textcolor{blue}{$\backslash$delta} \texttt{-  para a letra \delta}\\
\textcolor{blue}{$\backslash$epsilon} \texttt{-  para a letra \epsilon}&&\textcolor{blue}{$\backslash$varepsilon} \texttt{-  para a caracter \varepsilon}\\
\textcolor{blue}{$\backslash$zeta} \texttt{-  para a letra \zeta}&&\textcolor{blue}{$\backslash$eta} \texttt{-  para a letra \eta}\\
\textcolor{blue}{$\backslash$theta} \texttt{-  para a letra \theta}&&\textcolor{blue}{$\backslash$vartheta} \texttt{-  para o caracter \vartheta}\\
\textcolor{blue}{$\backslash$iota} \texttt{-  para a letra \iota}&&\textcolor{blue}{$\backslash$kappa} \texttt{-  para a letra \kappa}\\
\textcolor{blue}{$\backslash$lambda} \texttt{-  para a letra \lambda}&&\textcolor{blue}{$\backslash$mu} \texttt{-  para a letra \mu}\\
\textcolor{blue}{$\backslash$nu} \texttt{-  para a letra \nu}&&\textcolor{blue}{$\backslash$xi} \texttt{-  para a letra \xi}\\
\textcolor{blue}{o} \texttt{-  para a letra ômicron}&&\textcolor{blue}{$\backslash$pi} \texttt{-  para a letra \pi}\\
\textcolor{blue}{$\backslash$varpi} \texttt{-  para o caracter \varpi}&&\textcolor{blue}{$\backslash$rho} \texttt{-  para a letra \rho}\\
\textcolor{blue}{$\backslash$varrho} \texttt{-  para o caracter \varrho}&&\textcolor{blue}{$\backslash$sigma} \texttt{-  para a letra \sigma}\\
\textcolor{blue}{$\backslash$varsigma} \texttt{-  para o caracter \varsigma}&&\textcolor{blue}{$\backslash$tau} \texttt{-  para a letra \tau}\\
\textcolor{blue}{$\backslash$upsilon} \texttt{-  para a letra \upsilon}&&\textcolor{blue}{$\backslash$phi} \texttt{-  para a letra \phi}\\
\textcolor{blue}{$\backslash$varphi} \texttt{-  para o caracter \varphi}&&\textcolor{blue}{$\backslash$chi} \texttt{-  para a letra \chi}\\
\textcolor{blue}{$\backslash$psi} \texttt{-  para a letra \psi}&&\textcolor{blue}{$\backslash$omega} \texttt{-  para a letra \omega}
\end{tabular}\\[5pt]
Caracteres Maiúsculos\\
\begin{tabular}{lcl}
\textcolor{blue}{$\backslash$Gamma} \texttt{-  para a letra \Gamma}&&\textcolor{blue}{$\backslash$Delta} \texttt{-  para a letra \Delta}\\
\textcolor{blue}{$\backslash$Theta} \texttt{-  para a letra \Theta}&&\textcolor{blue}{$\backslash$Lambda} \texttt{-  para a letra \Lambda}\\
\textcolor{blue}{$\backslash$Xi} \texttt{-  para a letra \Xi}&&\textcolor{blue}{$\backslash$Pi} \texttt{-  para a letra \Pi}\\
\textcolor{blue}{$\backslash$Sigma} \texttt{-  para a letra \Sigma}&&\textcolor{blue}{$\backslash$Upsilon} \texttt{-  para a letra \Upsilon}\\
\textcolor{blue}{$\backslash$Phi} \texttt{-  para a letra \Phi}&&\textcolor{blue}{$\backslash$Psi} \texttt{-  para a letra \Psi}\\
\textcolor{blue}{$\backslash$Omega} \texttt{-  para a letra \Omega}&&\\
\end{tabular}
\end{tcolorbox}
}



\newcommand{\adicionaFig}{
\begin{tcolorbox}[colback=blue!5!white,colframe=green!75!blue,title= Comando Acrescenta figura, fontlower=\footnotesize]
Para adicionar uma figura no documento siga os passos abaixo:\\[5pt]
\textcolor{blue}{$\backslash$begin}\{figure\}[h!]\\
\texttt{Inicia o ambiente, o ``h!'' significa aqui, ``H'' - Ambos fixam a imagem naquela exata posição do texto}\\
\textcolor{blue}{$\backslash$setstretch\{1.5\}$\backslash$centering$\backslash$footnotesize} \\
\texttt{Configura Rodapé: espaçamento - centralização - tamanho da fonte}\\
\textcolor{blue}{$\backslash$caption}\{texto\}\texttt{ - ``texto'' é a descrição da figura no documento} \\
\textcolor{blue}{$\backslash$begin}\{center\} \texttt{centraliza a imagem}\\
\textcolor{blue}{$\backslash$includegraphics}[tamanho]\{nomeArquivo\}\\
\texttt{ - ``tamanho'' pode ser escalonado(ex. scale=0.5) ou largura(width) e altura(height), ex. width=3cm, height=0.5$\backslash$linewidth.\\
``nome.ext'' todas as imagens devem ser adicionadas na pasta ``imagens''. A ``.ext'' refere-se a extensão da imagem ex. .jpg,.png,etc}\\
\textcolor{blue}{$\backslash$end}\{center\} \texttt{Fim da centralização da Imagem}\\
\textcolor{blue}{$\backslash$label}\{fig:nomedafigura\}\texttt{ - indica como a figura será referenciada no documento}\\
\textcolor{blue}{$\backslash$par} Fonte:  local de origem \texttt{ - indica o local de origem da figura ou quem a produziu}\\
\textcolor{blue}{$\backslash$end}\{figure\}\texttt{ - finaliza o ambiente figure}
\tcblower
\textcolor{blue}{EXEMPLO}\\
$\backslash$begin\{figure\}[H]\\
\tab $\backslash$setstretch\{1.5\}$\backslash$centering$\backslash$footnotesize\\
\tab $\backslash$caption\{Album A Presenca da Gloria da banda Santa Geracao\}\\
\tab $\backslash$begin\{center\}\\
\tab \tab $\backslash$includegraphics[width=0.5$\backslash$textwidth]\{presencaDaGloria.png\}\\
\tab $\backslash$end\{center\}\\
\tab $\backslash$label\{fig:capaPresencaDaGloria\}\\
\tab $\backslash$par Fonte: Google Imagens\\
$\backslash$end\{figure\}\\
\end{tcolorbox}
}

\newcommand{\adicionaQuadro}{
\begin{tcolorbox}[colback=blue!5!white,colframe=green!75!blue,title= Comando Acrescenta Quadro, fontlower=\footnotesize]
Acesse o site: https://www.tablesgenerator.com para produzir um quadro/tabela\\
\textcolor{blue}{$\backslash$begin}\{Quadro\}[H]\texttt{ - inicia o ambiente quadros ``H'' aqui!}\\
\textcolor{blue}{$\backslash$setstretch\{1.5\}$\backslash$centering$\backslash$footnotesize} \\
\texttt{Conf. Rodapé: espaçamento|centralização|tamanho da fonte}\\
\textcolor{blue}{$\backslash$caption}\{texto\}\texttt{ - ``texto'' é a descrição do quadro}\\
\textcolor{blue}{$\backslash$begin}\{center\}\texttt{ - inicia o ambiente centralizar}\\
\textcolor{blue}{$\backslash$begin}\{tabular\}\{|c|c|\}\textcolor{blue}{$\backslash$hline}\texttt{- inicia o ambiente tabular que cria a estrutura do quadro, ``|c|c|'' são duas colunas centralizadas, ``$\backslash$hline'' insere uma linha horizontal}\\
Título Coluna1 \& Título Coluna2 \textcolor{blue}{$\backslash \backslash \backslash$hline}\texttt{ - Texto das colunas ``\&'' separa as colunas}\\
valor1 \& valor2 \textcolor{blue}{$\backslash\backslash \backslash$hline}\texttt{ - valores das colunas para uma linha. Para acrescentar mais linhas insira os valores, separados por \& e finalize com $\backslash\backslash \backslash$hline}\\
\textcolor{blue}{$\backslash$end}\{tabular\}\texttt{ - finaliza o ambiente tabular}\\
\textcolor{blue}{$\backslash$end}\{center\}\texttt{ - finaliza o ambiente centralizar}\\
\textcolor{blue}{$\backslash$label}\{qua:nome\}\texttt{ - insere um rótulo para a chamar o quadro}\\
 \textcolor{blue}{$\backslash$par} Fonte: Elaborado por \texttt{ - indica a fonte do quadro}\\
\textcolor{blue}{$\backslash$end}\{Quadro\}\texttt{ - finaliza o ambiente quadro}
\tcblower
\textcolor{blue}{EXEMPLO}\\
$\backslash$begin\{Quadro\}[H]\\
\tab $\backslash$setstretch\{1.5\}$\backslash$centering$\backslash$footnotesize\\
\tab $\backslash$caption\{Quadro sem sentido.\}\\
\tab $\backslash$begin\{center\}\\
\tab \tab $\backslash$begin\{tabular\}\{|c|c|\}\\
\tab \tab \tab $\backslash$hline\\
\tab \tab \tab Título Coluna \& Título Coluna $\backslash$$\backslash$\\
\tab \tab \tab $\backslash$hline\\
\tab \tab \tab X \& Y $\backslash$$\backslash$\\
\tab \tab \tab $\backslash$hline\\
\tab \tab $\backslash$end\{tabular\}\\
\tab $\backslash$end\{center\}\\
\tab $\backslash$label\{qua:tabela-ssentido\}\\
\tab $\backslash$par Fonte: Elaborado pelo próprio autor.\\
$\backslash$end\{Quadro\}\\
\end{tcolorbox}
}

\newcommand{\adicionaTabela}{
\begin{tcolorbox}[colback=blue!5!white,colframe=green!75!blue,title= Comando Adiciona Tabela, fontupper=\footnotesize, fontlower=\footnotesize]
Acesse o site: https://www.tablesgenerator.com para produzir um quadro/tabela\\
\textcolor{blue}{$\backslash$begin}\{table\}[H] \texttt{ - Inicia o ambiente tabela}\\
\textcolor{blue}{$\backslash$setstretch\{1.5\}$\backslash$centering$\backslash$footnotesize} \\
\texttt{Conf. Rodapé: espaçamento|centralização|tamanho da fonte}\\
\textcolor{blue}{$\backslash$caption}\{texto\}\texttt{ - ``texto'' é a descrição da tabela no documento}\\
\textcolor{blue}{$\backslash$begin}\{center\}\texttt{- inicia o ambiente centralizador da tabela na página}\\
\textcolor{blue}{$\backslash$begin}\{tabular\}\{p\{0.8cm\} crlS[table-format=0.1]\} \texttt{inicia o ambiente tabular}\\
\texttt{formato das colunas: ``p\{valor em \textit{cm}\}'' para coluna de tamanho fixo | ``c'' coluna com texto centralizado | ``r'' texto à direita | ``l'' texto à esquerda}\\
\textcolor{blue}{$\backslash$toprule}\texttt{ - linha superior da tabela}\\
Coluna1 \& coluna2 \textcolor{blue}{$\backslash\backslash$}\texttt{ - texto das colunas | ``\&'' separador dos dados das colunas | padrão usado no cabeçalho e dados}\\
\textcolor{blue}{$\backslash$midrule}\texttt{ - linha abaixo do cabeçalho}\\
\textcolor{blue}{$\backslash$bottomrule}\texttt{ - linha inferior(abaixo) da tabela}\\
\textcolor{blue}{$\backslash$end}\{tabular\}\texttt{ - finalização do ambiente tabular}\\
\textcolor{blue}{$\backslash$end}\{center\}\texttt{ - finalização do ambiente centralizador}\\
\textcolor{blue}{$\backslash$label}\{table:tabelanome\}\texttt{ - insere o rotulo ''table:tabelanome'' para chamar no documento}\\
\textcolor{blue}{$\backslash$par} Fonte: Elaborado por \texttt{ - indica a fonte dos dados da tabela}\\
\textcolor{blue}{$\backslash$end}\{table\}\texttt{ - finalização do ambiente tabela}
\tcblower
$\backslash$begin\{table\}[H]\\
\tab $\backslash$setstretch\{1.5\}$\backslash$centering$\backslash$footnotesize\\
\tab $\backslash$caption\{Cronograma\} \\
\tab $\backslash$begin\{center\} \\
\tab \tab $\backslash$begin\{tabular\}\{p\{7cm\}S[table-format=0.2]\}\\
\tab \tab \tab $\backslash$toprule \\
\tab \tab \tab \{2019\} \& \{Ago\} $\backslash$$\backslash$\\
\tab \tab \tab $\backslash$midrule\\
\tab \tab \tab Revisão bibliográfica  \& X \&  $\backslash$$\backslash$\\
\tab \tab \tab $\backslash$bottomrule\\
\tab \tab $\backslash$end\{tabular\} \\
\tab $\backslash$end\{center\} \\
\tab $\backslash$label\{table:cronograma\} \\
\tab $\backslash$par Fonte: Elaborado pelo próprio autor. \\
$\backslash$end\{table\} \\
\end{tcolorbox}
}

\newcommand{\upArquivos}{
\begin{tcolorbox}[colback=blue!5!white,colframe=green!75!blue,title= Ajuda para fazer Upload de arquivos]
Para fazer o upload de um arquivo siga os passos ilustrados abaixo:\\[5pt]
\ding{46} Na janela de arquivos do Overleaf, navegue até a pasta imagens \ding{202};\\
\ding{46} clique no icone de upload \ding{202};\\
\ding{46} navegue na pasta do computador até o local do arquivo \ding{203};\\
\ding{46} arraste o arquivo para a janela de upload \ding{204}.\\[5pt]
\begin{tikzpicture}
\node[] at (0.3, -5.8)  (t1) {\large\ding{202}};
\node[] at (5.3, -5.8)  (t3) {\large\ding{204}};
\node[] at (10.3, -5.8)  (t2) {\large\ding{203}};
\node[] at (0, -4)  (a1)    {\includegraphics[scale=0.3]{Backup/up1.png}}; 
\node[] at (1, -3.5)  (a2) {\includegraphics[scale=0.09]{Backup/up3a.png}};
\node[] at (5.5, -4)  (a3) {\includegraphics[scale=0.13]{Backup/up2.png}};
\node[] at (10.5, -4)  (a3) {\includegraphics[scale=0.15]{Backup/up4.png}};
\node[] at (8.5, -3)  (a3) {\includegraphics[scale=0.09]{Backup/up3b.png}};
\end{tikzpicture}
\end{tcolorbox}
}

\newcommand{\fazerLista}{
\begin{tcolorbox}[colback=blue!5!white,colframe=green!75!blue,title= Como criar uma lista numerada ou não]
Para criar uma Lista não numerada siga os passos abaixo:\\[5pt] 
\textcolor{blue}{$\backslash$begin}\{itemize\} \texttt{ - Inicia o ambiente lista}\\
\textcolor{blue}{$\backslash$item} primeiro\\ 
\textcolor{blue}{$\backslash$item} segundo\\
\textcolor{blue}{$\backslash$end}\{itemize\} \texttt{ - Finaliza o ambiente lista}\\[5pt]
\texttt{você deverá digitar o primeiro elemento da sua lista depois do comando  ``$\backslash$item''. Para acrescentar mais itens, após o ``Enter'' digite ``$\backslash$item'' antes do seu texto}
 \begin{multicols}{2}
Sua lista terá a aparência a seguir
\begin{itemize}
\item primeiro
\item segundo
\end{itemize}

\columnbreak
mudando o rótulo: \\\textcolor{blue}{$\backslash$begin}\{itemize\}[label=\textcolor{blue}{$\backslash$ding}\{42\}]
\begin{itemize}[label=\ding{42}]
\item primeiro
\item segundo
\end{itemize}
\end{multicols}
\textcolor{green!70!black}{Para mais rótulos, no Google procure \textit{pifont table}}\\[5pt]
Para criar uma Lista não numerada\\
\textcolor{blue}{$\backslash$begin}\{enumerate\} \texttt{ - Inicia o ambiente lista enumerada}\\
\textcolor{blue}{$\backslash$item} primeiro\\ 
\textcolor{blue}{$\backslash$item} segundo\\
\textcolor{blue}{$\backslash$end}\{enumerate\} \texttt{ - Finaliza o ambiente lista enumerada}
\begin{multicols}{2}
Sua lista terá a aparência a seguir
\begin{enumerate}
\item primeiro
\item segundo
\end{enumerate}

\columnbreak
mudando o rótulo: \\\textcolor{blue}{$\backslash$begin}\{enumerate\}[label=(\textcolor{blue}{$\backslash$alph*})]
\begin{enumerate}[label=(\alph*)]
\item primeiro
\item segundo
\end{enumerate}
\end{multicols}
\end{tcolorbox}
}

\newcommand{\adicionaCodigo}{
\begin{tcolorbox}[colback=blue!5!white,colframe=green!75!blue,title= Comando Adiciona Código, fontlower=\footnotesize]
Para adicionar código de uma linguagem de programação, siga os passos abaixo:\\[5pt] 
\textcolor{blue}{$\backslash$begin}\{lstlisting\}[language=\{\textcolor{blue}{NomeLinguagem}\}, caption=\{\textcolor{blue}{Nome Do Código}\}] \\ \texttt{ - Inicia o ambiente de código \\ Informa a Linguagem de Programação \\ Informa o título do código}\\
Código da Linguagem de Programação\\
\textcolor{blue}{$\backslash$end}\{lstlisting\}\texttt{ - finalização do ambiente de código}
\tcblower
\textcolor{blue}{EXEMPLO}\\
$\backslash$begin\{lstlisting\}[language=Python, caption=HelloWorld]\\
\tab print("Ola, Mundo")\\
$\backslash$end\{lstlisting\}\\
\end{tcolorbox}
}


%% \adicionaSigla
\newcommand{\adicionaSigla}{
\begin{tcolorbox}[colback=blue!5!white,colframe=green!75!blue,title= Comando Adiciona Sigla, fontlower=\footnotesize]
Para adicionar uma Sigla importante ao Texto e que deverá aparecer na \textcolor{blue}{Lista de Siglas e Abreviações}, siga os passos abaixo:\\[5pt] 
\textcolor{blue}{1 - Acesse o arquivo \{Escritor$\backslash$abreviaçõesSiglas.tex\}} \\[5pt]
\textcolor{blue}{EXEMPLO}\\[5pt]
\textcolor{blue}{$\backslash$DeclareAcronym}\{sbc\}\{\\
\texttt{Comando gerador de sigla \{ palavraChaveSigla \}}\\
\tab short = \normalfont{SBC}, \texttt{A Sigla como aparecerá}\\
\tab short-plural = s, \texttt{Sigla no modo plural (adição de s)}\\
\tab long  = Sociedade Brasileira de Computação, \texttt{por Extenso}\\
\tab tag = abbrev \texttt{abbrev - indicativo de SIGLA}\\
\} \texttt{- Fim da declaração de Sigla}\\[5pt]
\textcolor{blue}{2 - Faça referência à Sigla no Texto}
\tcblower
\textcolor{blue}{EXEMPLO}\\
\texttt{Teremos 2 comandos}\\[5pt]
\textcolor{blue}{$\backslash$ac}\{palavraChaveSigla\}\\
\texttt{Usado para mostrar a SIGLA literal}\\
\textcolor{blue}{$\backslash$acl}\{palavraChaveSigla\}\\
\texttt{Usado para mostrar a forma extensa da Sigla}\\
\textcolor{blue}{EXEMPLO}\\[5pt]
$\backslash$ac\{sbc\}\\
$\backslash$acl\{sbc\}\\
\\[5pt]
\textcolor{blue}{SBC}\\
\textcolor{blue}{Sociedade Brasileira de Computação}
\end{tcolorbox}
}


%% \adicionaSimbolo
\newcommand{\adicionaSimbolo}{
\begin{tcolorbox}[colback=blue!5!white,colframe=green!75!blue,title= Comando Adiciona Símbolo, fontlower=\footnotesize]
Para adicionar um Símbolo importante ao texto e que deverá aparecer na \textcolor{blue}{Lista de Símbolos}, siga os passos abaixo:\\[5pt] 
\textcolor{blue}{1 - Acesse o arquivo \{Escritor$\backslash$simbolos.tex\}} \\[5pt]
\textcolor{blue}{EXEMPLO}\\[5pt]
\textcolor{blue}{$\backslash$DeclareAcronym}\{megaByte\}\{\\
\texttt{Comando gerador de símbolo \{ palavraChaveSigla \}}\\
\tab short = \normalfont{MB}, \texttt{O Símbolo como aparecerá}\\
\tab long  = é uma unidade de medida de informação,\\
\tab \texttt{Significado do Símbolo por Extenso}\\
\tab sort  = MB, \texttt{Letra usada para ORDENAR na lista}\\
\tab tag = nomen \texttt{nomen- indicativo de SÍMBOLO}\\
\} \texttt{- Fim da declaração de Símbolo}\\[5pt]
\textcolor{blue}{2 - Faça referência ao Símbolo no Texto}
\tcblower
\textcolor{blue}{EXEMPLO}\\
\texttt{Teremos 2 comandos}\\[5pt]
\textcolor{blue}{$\backslash$ac}\{palavraChaveSigla\}\\
\texttt{Usado para mostrar a SIGLA literal}\\
\textcolor{blue}{$\backslash$acl}\{palavraChaveSigla\}\\
\texttt{Usado para mostrar a forma extensa da Sigla}\\
\textcolor{blue}{EXEMPLO}\\[5pt]
$\backslash$ac\{megaByte\}\\
$\backslash$acl\{megaByte\}\\
\\[5pt]
\textcolor{blue}{MB}\\
\textcolor{blue}{é uma unidade de medida de informação}
\end{tcolorbox}
}

%% \adicionaPalavraGlossario
\newcommand{\adicionaPalavraGlossario}{
\begin{tcolorbox}[colback=blue!5!white,colframe=green!75!blue,title= Comando Adiciona Palavras ao Glossário, fontlower=\footnotesize]
Para adicionar uma Palavra que é importante ao texto e que deverá aparecer no \textcolor{blue}{Glossário}, siga os passos abaixo:\\[5pt] 
\textcolor{blue}{1 - Acesse o arquivo \{Escritor$\backslash$glossario.tex\}} \\[5pt]
\textcolor{blue}{EXEMPLO}\\[5pt]
\textcolor{blue}{$\backslash$newglossaryentry}\{latex\}\{\\
\texttt{Gerador de Palavra do Glossário \{ palavraChaveGlossario \}}\\
\tab name = \{LATEX\}, \texttt{- Palavra como aparecerá no Glossário}\\
\tab text  = \{LaTeX\}, \texttt{- Como aparecerá no texto} \\
\tab sort = \{latex\}, \texttt{- Usado para Ordenar as palavra }\\
\tab description = \{É uma linguagem de marcação especialmente feita para documentos científicos\}, \\
\tab \texttt{Descrição por extenso da palavra - LaTeX}\\
\} \texttt{- Fim da declaração de Palavra do Glossário}\\[5pt]
\textcolor{blue}{2 - Faça referência à Palavra do Glossário no Texto} \\[5pt]
\textcolor{blue}{EXEMPLO}\\[5pt]
\texttt{Teremos 1 comando}\\[5pt]
\textcolor{blue}{$\backslash$gls}\{palavraChaveGlossário\}\\
\texttt{Usado para mostrar o texto da Palavra do Glossário}
\tcblower
\textcolor{blue}{EXEMPLO}\\
$\backslash$gls\{latex\}\\
\\[5pt]
\textcolor{blue}{LaTeX}\\
\end{tcolorbox}
}

%8888888888888888888888888888888888888888888888888888



% -------------------------------------------------
% Pacotes para QUDROS
% -------------------------------------------------
\newcommand{\quadroname}{Quadro}
\newcommand{\listofquadrosname}{Lista de Quadros}
\newfloat{Quadro}{\quadroname}{loq}[chapter]
\restylefloat*{Quadro}
\setfloatadjustment{Quadro}{\centering}
\setfloatlocations{Quadro}{hbtp}
\newlistof{listofquadros}{loq}{\listofquadrosname}
\newlistentry{Quadro}{loq}{0}
\counterwithout{Quadro}{chapter}
\renewcommand{\cftQuadroname}{\quadroname\space}
\renewcommand*{\cftQuadroaftersnum}{\hfill\textendash\hfill}


% -------------------------------------------------
% Configurações para aumentar a quantidade caracteres
% -------------------------------------------------
\tolerance=1
\emergencystretch=\maxdimen
\hyphenpenalty=10000
\hbadness=10000

% -------------------------------------------------
% Configurações para definir um comando de espaço inical no parágrafo
% -------------------------------------------------
\newcommand\tab[1][1cm]{\hspace*{#1}}   
%%%%%       \tab

% -------------------------------------------------
% Configurações para Definir quais capítulos aparecerão
% -------------------------------------------------
\providecommand{\Dedicatoria}{}
\newcommand{\temDedicatoria}[1]{
\renewcommand{\Dedicatoria}{#1}
}

\providecommand{\Agradecimento}{}
\newcommand{\temAgradecimento}[1]{
\renewcommand{\Agradecimento}{#1}
}

\providecommand{\Epigrafe}{}
\newcommand{\temEpigrafe}[1]{
\renewcommand{\Epigrafe}{#1}
}

\providecommand{\Ilustracoes}{}
\newcommand{\temListadeIlustracoes}[1]{
\renewcommand{\Ilustracoes}{#1}
}

\providecommand{\Tabelas}{}
\newcommand{\temListadeTabelas}[1]{
\renewcommand{\Tabelas}{#1}
}

\providecommand{\LCodigos}{}
\newcommand{\temListadeCodigos}[1]{
\renewcommand{\LCodigos}{#1}
}

\providecommand{\LQuadros}{}
\newcommand{\temListadeQuadros}[1]{
\renewcommand{\LQuadros}{#1}
}

\providecommand{\Siglas}{}
\newcommand{\temListadeSiglas}[1]{
\renewcommand{\Siglas}{#1}
}

\providecommand{\Simbolos}{}
\newcommand{\temListadeSimbolos}[1]{
\renewcommand{\Simbolos}{#1}
}

\providecommand{\Glossario}{}
\newcommand{\temGlossario}[1]{
\renewcommand{\Glossario}{#1}
}

\providecommand{\Apendices}{}
\newcommand{\temApendices}[1]{
\renewcommand{\Apendices}{#1}
}

\providecommand{\Anexos}{}
\newcommand{\temAnexos}[1]{
\renewcommand{\Anexos}{#1}
}

\providecommand{\Indice}{}
\newcommand{\temIndice}[1]{
\renewcommand{\Indice}{#1}
}

\providecommand{\ExemploTexto}{}
\newcommand{\temExemploTexto}[1]{
\renewcommand{\ExemploTexto}{#1}
}

\newenvironment{citacaodireta}
{\begin{quotation}\fontsize{10}{10}\selectfont}%\footnotesize}%
{\end{quotation}}


%marca{X} para Sim, caso contrário {}
%PRÉ TEXTUAIS
\temDedicatoria{x}
\temAgradecimento{x}
\temEpigrafe{x}
\temListadeCodigos{x}
\temListadeQuadros{x}
\temListadeTabelas{x}
\temListadeIlustracoes{x}
\temListadeSiglas{x}
\temListadeSimbolos{x}

%PÓS TEXTUAIS
\temGlossario{x}
\temApendices{x}
\temAnexos{x}
\temIndice{x}

%BACKUP
\temExemploTexto{x}

\makeindex