%NBR 6028 http://plone.ufpb.br/secretariado/contents/documentos/2021_ABNT6028Resumo.pdf
%4.1.8 Quanto à sua extensão, convém que os resumos tenham:
%a) 150 a 500 palavras nos trabalhos acadêmicos e relatórios técnicos e/ou científicos;
%b) 100 a 250 palavras nos artigos de periódicos;
%c) 50 a 100 palavras nos documentos não contemplados nas alíneas anteriores.

O resumo deve apresentar de forma concisa os pontos relevantes de um texto, fornecendo uma visão rápida e clara do conteúdo e das conclusões do trabalho. O texto, redigido na forma impessoal do verbo, é constituído de uma sequência de frases concisas e objetivas e não de uma simples enumeração de tópicos, não ultrapassando 500 palavras, seguido, logo abaixo, das palavras representativas do conteúdo do trabalho, isto é, palavras-chave e/ou descritores. Por exemplo, deve-se evitar, na redação do resumo, o uso de fórmulas, equações, diagramas e símbolos, optando-se, quando necessário, pela transcrição na forma extensa, além de não incluir citações bibliográficas.  Citação no rodapé normalmente utilizada para referenciar páginas Web.

\palavraschave{Palavra-chave 1; Palavra-chave 2; Palavra-chave 3}

Palavras-chave: \imprimirpalavraschave .