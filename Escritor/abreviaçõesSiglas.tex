%%%  LISTA DE ABREVIAÇÕES e SIMBOLOS
% no decorrer da sua escrita, você usará abreviações e símbolos
% estes simbolos e abreviações devem constar na lista de abreviações
% e simbolos. Para gerar AUTOMATICAMENTE estas listas sem o menor esforço 
% você deve declarar NESTE documento todas as ABREVIAÇÕES e todos os SÍMBOLOS
% que você usará no decorrer do texto. 
% 
% SOMENTE DEPOIS DE DECLARAR AQUI
%
% você será capaz de pôr o simbolo usando a biblioteca ACRONYM
%
% usando os comandos        \ac      e     \acl
%	
%	 Para utilizar abreviaturas e simbolos, basta utilizar o comando barra \ combinado com ac ou acl.
%	 
%	 Abreviaturas e a versão por extenso.
%	 
%	 AC \ac{abntex}.     MOSTRARÁ A ABREVIAÇÃO
%%	 
%	 ACL \acl{abntex}.   MOSTRARÁ AS PALAVRAS DA ABREVIAÇÃO POR EXTENSO
%	 
%%	 Simbolos e a versão por extenso.
%	 
%	 AC  \ac{lambda}.   MOSTRARÁ O SIMBOLO
%	 
%	 ACL \acl{lambda}.  MOSTRARÁ A PALAVRA DO SIMBOLO POR EXTENSO
% 
% class `abbrev': abbreviations:
%NOTE QUE SOMENTE OS QUE SÃO CHAMADOS NO DECORRER DO TEXTO IRÃO APARECER NA LISTA DE ABREVIATURAS E SIMBOLOS
%PADRÃO PARA ABREVIATURA

%\DeclareAcronym{}{
%	short = \normalfont{},
%	long  = \textit{},
%	class = abbrev
%}

\DeclareAcronym{sbc}{
	short = \normalfont{SBC},
	short-plural = s,
	long  = Sociedade Brasileira de Computação,
	tag = abbrev
}

\DeclareAcronym{nc}{
	short = \normalfont{NC},
	long  = \textit{Networking Coding},
	tag = abbrev
}

\DeclareAcronym{abnt}{
	short = \normalfont{ABNT},
	short-plural = s,
	long  = Associação Brasileira de Normas Técnicas,
	tag = abbrev
}

\DeclareAcronym{abntex}{
	short = \normalfont{abnTeX},
	short-plural = s,
	long  = ABsurdas Normas para TeX,
	tag = abbrev
}

