% PADRÃO DE ENTRADA NO GLOSSÁRIO
%\newglossaryentry{aPalavra}{
%        name= {APALVRA},       %%% EM CAIXA ALTA, É O QUE IRÁ APARECER NO GLOSSÁRIO, PARA SEGUIR A NORMA DO IFPA
%        text = {aPalavra},    %%% como irá aparecer no TEXTO
%        sort = {Formula},     %% REESCREVA A PALAVRA SEM USAR NENHUMA ACENTUAÇÃO PARA QUE POSSA SER USADA PARA ORGANZIAR O GLOSSÁRIO
%        description={É uma linguagem de marcação especialmente feita para documentos científicos}              %DESCRIÇÃO
%}
%
% ADICIONE AS ENTRADAS USANDO A ORDEM ALFABÉTICA
%

\newglossaryentry{fórmula}{
        name = {FÓRMULA},
        text = {fórmula},
        sort = {Formula},     %% PALAVRA ACENTUADAS DEVERÃO CONTER O CAMPO 'SORT' PARA AJUDAR NA ORDENAÇÃO 
        description={Uma expressão matemática}
}

\newglossaryentry{latex}{
        name= {LATEX},
        text = {latex},
        sort = {latex},
        description={É uma linguagem de marcação especialmente feita para documentos científicos}
}

\newglossaryentry{matemática}{
        name = {MATEMÁTICA},
        text = {matemática},
        sort = {matematica},
        description={Matemática é o que a matemática faz}
}

\longnewglossaryentry{sample}{
    name = {SAMPLE},
    text = {sample},            % -    versão em texto
    plural = {samples},         %  -    versão em plural no texto
    sort = {sample},
}{%
     {an example}       %DESCRIÇÃO
}

%\newterm[〈key=value list〉]{〈entry-label〉}
%This is just a shortcut that uses \newglossaryentry with the name set to 〈entry-label〉 and
%the description is suppressed.


%\newglossaryentry{not:set}
%{% This entry goes in the `notation' glossary:
%type={notation},
%name={$\mathcal{S}$},
%text={\mathcal{S}},
%description={A set},
%sort={S}
%}